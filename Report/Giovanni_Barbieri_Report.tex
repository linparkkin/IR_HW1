\documentclass[a4paper, 11pt]{article}
\usepackage[utf8x]{inputenc}
\usepackage{comment} % enables the use of multi-line comments (\ifx \fi) 
\usepackage{fullpage} % changes the margin
\usepackage{graphicx}
\usepackage{hyperref}

\usepackage[
singlelinecheck=false % <-- important
]{caption}

\begin{document}
%Header-Make sure you update this information!!!!
\noindent
\large\textbf{Homework 1} \\
\normalsize \textit{Studente:} Giovanni Barbieri\\
\textit{Matricola:} 1177495\\
\textit{Repository:} \url{https://github.com/linparkkin/IR_HW1}


\section*{Introduzione}
Lo svolgimento del seguente homework consiste nell'analisi della collezione sperimentale \textit{TREC7} composta da circa 52800 documenti, 50 topic e un pool con due gradi di rilevanza: R, NR. \\
L'analisi consiste nell'utilizzo di uno strumento di Information Retrieval, in questo caso \textit{terrier v4.4}, per lo sviluppo di 4 diverse run: 
\begin{itemize}
 \item stoplist, porter stemmer, BM25;
 \item no stoplist, porter stemmer, BM25;
 \item stoplist, porter stemmer, TF*IDF;
 \item no stoplist, no porter stemmer, TF*IDF;
\end{itemize}
Valutare poi le run calcolando MAP, Rprec e Precision at 10 utilizzando un tool di valutazione, in questo caso \textit{trec\_eval}, già disponibile all'interno di terrier. \\
Condurre alla fine il test statistico  ANOVA 1-way per determinare i sistemi appartenenti al ``top group'' sulla base delle diverse misure. 

\section*{Valutazione}
Prima di tutto è stato utilizzato $terrier$ per effetuare l'indicizzazione dei documenti presenti nella collezione. Per fare ciò è stato modificato il file $terrier.properties$ del tool, cambiando di volta in volta in volta i parametri in base alla run che si era interessati a sviluppare. Va specificato che nella configurazione utilizzata in questo homework sono stati presi in considerazione anche i termini con un basso indice di frequenza, $ignore.low.idf.terms=false$. \\
Una volta ottenuti i 4 diversi indici è stato possibile valutare le run utilizzando $trec\_eval$ dal terminale, lanciando il comando \texttt{sh trec\_eval.sh -q -m map -m Rprec -m P.10 qrels.trec7.txt}. Non si è tenuto conto della descrizione dei topic settando $TrecQueryTags.skip=DESC,NARR$.
La seguente tabella riassume i risultati ottenuti.

{%
\newcommand{\mc}[3]{\multicolumn{#1}{#2}{#3}}
\begin{center}
    \begin{tabular}{l|l|l|l|}\cline{2-4}
    & \textbf{map} & \textbf{Rprec} & \textbf{P@10}\\\hline
    \mc{1}{|l|}{stoplist, porter stemmer, BM25} & 0.1828 & 0.2391 & 0.4180\\\hline
    \mc{1}{|l|}{no stoplist, porter stemmer, BM25} & 0.1854 & 0.2406 & 0.4300\\\hline
    \mc{1}{|l|}{stoplist, porter stemmer, TF*IDF} & 0.1821 & 0.2391 & 0.4200\\\hline
    \mc{1}{|l|}{no stoplist, no porter stemmer, TF*IDF} & 0.1693 & 0.2290 & 0.4060\\\hline
    \end{tabular}
\end{center}
}%




\end{document} 
